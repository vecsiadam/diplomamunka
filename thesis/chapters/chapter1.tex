% !TEX encoding = UTF-8 Unicode

\Chapter{Bevezetés}
A mai világban az egyik legismertebb kommunikációs forma a chat-elés. Használjuk munkahelyen és otthon is, akár számítógépen, mobiltelefonon, tábla gépen vagy okos órán. A Számítógépen talán a legismertebb chat alkalmazások a Skype és a Slack, de a legtöbb közösségi média platform lehetővé teszi számunkra ezt a kommunikációs eszközt. Az Instagram, Facebook, Snapchat, LinkedIn is ad ilyen lehetőséget így a barátainkkal, ismerőseinkkel tudunk csevegni. De nem csak magán jellegű beszélgetésekre alkalmas ez, sokan használjuk munkahelyen is információ csere céljából.

Legnagyobb előnye, hogy valós időben történik az üzenetváltás bizonyos esetekben akár kép és videó megosztás, fájl csere is.  Ezen kívül ami fontos még a chat alkalmazásokban, hogy az üzenetek naplózva lesznek, így korábbi beszélgetéseket is vissza lehet olvasni és nem vész el az információ. Ezen kívül általában keresni is lehet a beszélgetésekben.

Dolgozatomban egy chat alkalmazás működését szeretném bemutatni, hogyan jutnak el az üzenetek a küldőtől a célig, hogyan történik az üzenetek naplózása, valamint az üzenetekben való keresés.  Szeretnék készíteni egy chat alkalmazást egy oktatást segítő alkalmazás kiegészítéseként, melyben pontosan az előbb említett funkciókat valósítanám meg. Backend oldalon Java, Spring boot, Apache Kafka és Elastic search segítségével.
